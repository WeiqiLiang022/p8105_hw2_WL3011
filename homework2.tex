% Options for packages loaded elsewhere
\PassOptionsToPackage{unicode}{hyperref}
\PassOptionsToPackage{hyphens}{url}
%
\documentclass[
]{article}
\usepackage{amsmath,amssymb}
\usepackage{iftex}
\ifPDFTeX
  \usepackage[T1]{fontenc}
  \usepackage[utf8]{inputenc}
  \usepackage{textcomp} % provide euro and other symbols
\else % if luatex or xetex
  \usepackage{unicode-math} % this also loads fontspec
  \defaultfontfeatures{Scale=MatchLowercase}
  \defaultfontfeatures[\rmfamily]{Ligatures=TeX,Scale=1}
\fi
\usepackage{lmodern}
\ifPDFTeX\else
  % xetex/luatex font selection
\fi
% Use upquote if available, for straight quotes in verbatim environments
\IfFileExists{upquote.sty}{\usepackage{upquote}}{}
\IfFileExists{microtype.sty}{% use microtype if available
  \usepackage[]{microtype}
  \UseMicrotypeSet[protrusion]{basicmath} % disable protrusion for tt fonts
}{}
\makeatletter
\@ifundefined{KOMAClassName}{% if non-KOMA class
  \IfFileExists{parskip.sty}{%
    \usepackage{parskip}
  }{% else
    \setlength{\parindent}{0pt}
    \setlength{\parskip}{6pt plus 2pt minus 1pt}}
}{% if KOMA class
  \KOMAoptions{parskip=half}}
\makeatother
\usepackage{xcolor}
\usepackage[margin=1in]{geometry}
\usepackage{color}
\usepackage{fancyvrb}
\newcommand{\VerbBar}{|}
\newcommand{\VERB}{\Verb[commandchars=\\\{\}]}
\DefineVerbatimEnvironment{Highlighting}{Verbatim}{commandchars=\\\{\}}
% Add ',fontsize=\small' for more characters per line
\usepackage{framed}
\definecolor{shadecolor}{RGB}{248,248,248}
\newenvironment{Shaded}{\begin{snugshade}}{\end{snugshade}}
\newcommand{\AlertTok}[1]{\textcolor[rgb]{0.94,0.16,0.16}{#1}}
\newcommand{\AnnotationTok}[1]{\textcolor[rgb]{0.56,0.35,0.01}{\textbf{\textit{#1}}}}
\newcommand{\AttributeTok}[1]{\textcolor[rgb]{0.13,0.29,0.53}{#1}}
\newcommand{\BaseNTok}[1]{\textcolor[rgb]{0.00,0.00,0.81}{#1}}
\newcommand{\BuiltInTok}[1]{#1}
\newcommand{\CharTok}[1]{\textcolor[rgb]{0.31,0.60,0.02}{#1}}
\newcommand{\CommentTok}[1]{\textcolor[rgb]{0.56,0.35,0.01}{\textit{#1}}}
\newcommand{\CommentVarTok}[1]{\textcolor[rgb]{0.56,0.35,0.01}{\textbf{\textit{#1}}}}
\newcommand{\ConstantTok}[1]{\textcolor[rgb]{0.56,0.35,0.01}{#1}}
\newcommand{\ControlFlowTok}[1]{\textcolor[rgb]{0.13,0.29,0.53}{\textbf{#1}}}
\newcommand{\DataTypeTok}[1]{\textcolor[rgb]{0.13,0.29,0.53}{#1}}
\newcommand{\DecValTok}[1]{\textcolor[rgb]{0.00,0.00,0.81}{#1}}
\newcommand{\DocumentationTok}[1]{\textcolor[rgb]{0.56,0.35,0.01}{\textbf{\textit{#1}}}}
\newcommand{\ErrorTok}[1]{\textcolor[rgb]{0.64,0.00,0.00}{\textbf{#1}}}
\newcommand{\ExtensionTok}[1]{#1}
\newcommand{\FloatTok}[1]{\textcolor[rgb]{0.00,0.00,0.81}{#1}}
\newcommand{\FunctionTok}[1]{\textcolor[rgb]{0.13,0.29,0.53}{\textbf{#1}}}
\newcommand{\ImportTok}[1]{#1}
\newcommand{\InformationTok}[1]{\textcolor[rgb]{0.56,0.35,0.01}{\textbf{\textit{#1}}}}
\newcommand{\KeywordTok}[1]{\textcolor[rgb]{0.13,0.29,0.53}{\textbf{#1}}}
\newcommand{\NormalTok}[1]{#1}
\newcommand{\OperatorTok}[1]{\textcolor[rgb]{0.81,0.36,0.00}{\textbf{#1}}}
\newcommand{\OtherTok}[1]{\textcolor[rgb]{0.56,0.35,0.01}{#1}}
\newcommand{\PreprocessorTok}[1]{\textcolor[rgb]{0.56,0.35,0.01}{\textit{#1}}}
\newcommand{\RegionMarkerTok}[1]{#1}
\newcommand{\SpecialCharTok}[1]{\textcolor[rgb]{0.81,0.36,0.00}{\textbf{#1}}}
\newcommand{\SpecialStringTok}[1]{\textcolor[rgb]{0.31,0.60,0.02}{#1}}
\newcommand{\StringTok}[1]{\textcolor[rgb]{0.31,0.60,0.02}{#1}}
\newcommand{\VariableTok}[1]{\textcolor[rgb]{0.00,0.00,0.00}{#1}}
\newcommand{\VerbatimStringTok}[1]{\textcolor[rgb]{0.31,0.60,0.02}{#1}}
\newcommand{\WarningTok}[1]{\textcolor[rgb]{0.56,0.35,0.01}{\textbf{\textit{#1}}}}
\usepackage{longtable,booktabs,array}
\usepackage{calc} % for calculating minipage widths
% Correct order of tables after \paragraph or \subparagraph
\usepackage{etoolbox}
\makeatletter
\patchcmd\longtable{\par}{\if@noskipsec\mbox{}\fi\par}{}{}
\makeatother
% Allow footnotes in longtable head/foot
\IfFileExists{footnotehyper.sty}{\usepackage{footnotehyper}}{\usepackage{footnote}}
\makesavenoteenv{longtable}
\usepackage{graphicx}
\makeatletter
\def\maxwidth{\ifdim\Gin@nat@width>\linewidth\linewidth\else\Gin@nat@width\fi}
\def\maxheight{\ifdim\Gin@nat@height>\textheight\textheight\else\Gin@nat@height\fi}
\makeatother
% Scale images if necessary, so that they will not overflow the page
% margins by default, and it is still possible to overwrite the defaults
% using explicit options in \includegraphics[width, height, ...]{}
\setkeys{Gin}{width=\maxwidth,height=\maxheight,keepaspectratio}
% Set default figure placement to htbp
\makeatletter
\def\fps@figure{htbp}
\makeatother
\setlength{\emergencystretch}{3em} % prevent overfull lines
\providecommand{\tightlist}{%
  \setlength{\itemsep}{0pt}\setlength{\parskip}{0pt}}
\setcounter{secnumdepth}{-\maxdimen} % remove section numbering
\usepackage{booktabs}
\usepackage{longtable}
\usepackage{array}
\usepackage{multirow}
\usepackage{wrapfig}
\usepackage{float}
\usepackage{colortbl}
\usepackage{pdflscape}
\usepackage{tabu}
\usepackage{threeparttable}
\usepackage{threeparttablex}
\usepackage[normalem]{ulem}
\usepackage{makecell}
\usepackage{xcolor}
\ifLuaTeX
  \usepackage{selnolig}  % disable illegal ligatures
\fi
\usepackage{bookmark}
\IfFileExists{xurl.sty}{\usepackage{xurl}}{} % add URL line breaks if available
\urlstyle{same}
\hypersetup{
  pdftitle={p8105\_hw2\_WL3011},
  pdfauthor={Weiqi Liang},
  hidelinks,
  pdfcreator={LaTeX via pandoc}}

\title{p8105\_hw2\_WL3011}
\author{Weiqi Liang}
\date{2024-10-02}

\begin{document}
\maketitle

\subsection{Setup File}\label{setup-file}

\begin{Shaded}
\begin{Highlighting}[]
\FunctionTok{library}\NormalTok{(tidyverse)}
\FunctionTok{library}\NormalTok{(dplyr)}
\FunctionTok{library}\NormalTok{(readxl)}
\FunctionTok{library}\NormalTok{(haven)}
\end{Highlighting}
\end{Shaded}

\subsection{I. Problem 1}\label{i.-problem-1}

\subsubsection{1.1 Load the NYC Subway
Dataset}\label{load-the-nyc-subway-dataset}

\begin{itemize}
\tightlist
\item
  Retained columns: \texttt{Line}, \texttt{Station\ Name},
  \texttt{Station\ Latitude}, \texttt{Station\ Longitude},
  \texttt{Route1}:\texttt{Route11}, \texttt{Entry}, \texttt{Vending},
  \texttt{Entrance\ Type}, \texttt{ADA};
\item
  Converted the \texttt{Entry} variable from ``YES'' and ``NO'' to
  logical TRUE and FALSE.
\end{itemize}

\begin{Shaded}
\begin{Highlighting}[]
\CommentTok{\# Load the NYC Subway csv}
\NormalTok{subway\_df }\OtherTok{=} 
  \FunctionTok{read.csv}\NormalTok{(}\StringTok{"./NYC\_Transit\_Subway\_Entrance\_And\_Exit\_Data.csv"}\NormalTok{, }
           \AttributeTok{na =} \FunctionTok{c}\NormalTok{(}\StringTok{"NA"}\NormalTok{, }\StringTok{"."}\NormalTok{, }\StringTok{""}\NormalTok{)) }\SpecialCharTok{|\textgreater{}}
\NormalTok{  janitor}\SpecialCharTok{::}\FunctionTok{clean\_names}\NormalTok{() }\SpecialCharTok{|\textgreater{}}
  \FunctionTok{select}\NormalTok{(line, station\_name, station\_latitude, station\_longitude,}
\NormalTok{         route1}\SpecialCharTok{:}\NormalTok{route11, entry, vending, entrance\_type, ada) }\SpecialCharTok{|\textgreater{}}
  \FunctionTok{mutate}\NormalTok{(}
    \AttributeTok{entry\_logical =} \FunctionTok{case\_match}\NormalTok{(}
\NormalTok{      entry, }
      \StringTok{"YES"} \SpecialCharTok{\textasciitilde{}} \ConstantTok{TRUE}\NormalTok{,}
      \StringTok{"NO"}  \SpecialCharTok{\textasciitilde{}} \ConstantTok{FALSE}
\NormalTok{      )}
\NormalTok{    )}
\end{Highlighting}
\end{Shaded}

The dataset now is tidy, with each row representing a station entrance
and all columns having consistent data types. It has 1868 rows and 20
columns, showing New York City subway station entrances/exits. The
columns include:

\textbf{Station details}: Line, Station Name

\textbf{Routes}: Route1 to Route11, showing the subway lines served at
the station

\textbf{Facilities}: Entrance Type, Entry, Vending, ADA (compliance),
and ADA Notes

\textbf{Geographical details}: Entrance Latitude/Longitude, North/South
and East/West Streets

\subsubsection{1.2 Answering the Following
Question}\label{answering-the-following-question}

\begin{enumerate}
\def\labelenumi{\arabic{enumi}.}
\tightlist
\item
  How many distinct stations are there?
\end{enumerate}

\begin{Shaded}
\begin{Highlighting}[]
\CommentTok{\# calculate the number of distinct stations (identified by name and line)}
\NormalTok{distinct\_stations }\OtherTok{=}\NormalTok{ subway\_df }\SpecialCharTok{|\textgreater{}}
  \FunctionTok{distinct}\NormalTok{(station\_name, line) }\SpecialCharTok{|\textgreater{}}
  \FunctionTok{nrow}\NormalTok{()}
\end{Highlighting}
\end{Shaded}

There are 465 distinct stations, identified by their name and line.

\begin{enumerate}
\def\labelenumi{\arabic{enumi}.}
\setcounter{enumi}{1}
\tightlist
\item
  How many stations are ADA compliant?
\end{enumerate}

\begin{Shaded}
\begin{Highlighting}[]
\CommentTok{\# the number of ADA compliant stations}
\NormalTok{ada\_compliant\_stations }\OtherTok{=}\NormalTok{ subway\_df }\SpecialCharTok{|\textgreater{}}
  \FunctionTok{filter}\NormalTok{(ada }\SpecialCharTok{==} \ConstantTok{TRUE}\NormalTok{) }\SpecialCharTok{|\textgreater{}}
  \FunctionTok{distinct}\NormalTok{(station\_name, line) }\SpecialCharTok{|\textgreater{}}
  \FunctionTok{nrow}\NormalTok{()}
\end{Highlighting}
\end{Shaded}

There are 84 ADA compliant stations.

\begin{enumerate}
\def\labelenumi{\arabic{enumi}.}
\setcounter{enumi}{2}
\tightlist
\item
  What proportion of station entrances / exits without vending allow
  entrance?
\end{enumerate}

\begin{Shaded}
\begin{Highlighting}[]
\NormalTok{no\_vending\_entry }\OtherTok{=}\NormalTok{ subway\_df }\SpecialCharTok{|\textgreater{}}
  \FunctionTok{filter}\NormalTok{(vending }\SpecialCharTok{==} \StringTok{"NO"}\NormalTok{) }\SpecialCharTok{|\textgreater{}}
  \FunctionTok{filter}\NormalTok{(entry }\SpecialCharTok{==} \StringTok{"YES"}\NormalTok{ ) }\SpecialCharTok{|\textgreater{}}
  \FunctionTok{nrow}\NormalTok{()}
\NormalTok{proportion }\OtherTok{=}\NormalTok{ no\_vending\_entry}\SpecialCharTok{/}\FunctionTok{nrow}\NormalTok{(subway\_df)}
\end{Highlighting}
\end{Shaded}

The proportion of station entrances without vending allow entrance is
0.0369379.

\subsubsection{1.3 Reformat Dataset}\label{reformat-dataset}

\begin{enumerate}
\def\labelenumi{\arabic{enumi}.}
\tightlist
\item
  Reformat data so that route number and route name are distinct
  variables.
\end{enumerate}

\begin{Shaded}
\begin{Highlighting}[]
\CommentTok{\# Split route to long format (pivot\_longer)}
\NormalTok{Reformat\_subway\_df }\OtherTok{=}\NormalTok{ subway\_df }\SpecialCharTok{|\textgreater{}}
  \FunctionTok{mutate\_at}\NormalTok{(}\FunctionTok{vars}\NormalTok{(route1}\SpecialCharTok{:}\NormalTok{route11), as.character) }\SpecialCharTok{|\textgreater{}}
  \FunctionTok{pivot\_longer}\NormalTok{(}
    \AttributeTok{cols =}\NormalTok{ route1}\SpecialCharTok{:}\NormalTok{route11, }
    \AttributeTok{names\_to =} \StringTok{"route\_number"}\NormalTok{, }
    \AttributeTok{values\_to =} \StringTok{"route\_name"}
\NormalTok{    ) }\SpecialCharTok{|\textgreater{}}
  \FunctionTok{filter}\NormalTok{(}\SpecialCharTok{!}\FunctionTok{is.na}\NormalTok{(route\_name))  }\CommentTok{\# remove NA}

\FunctionTok{head}\NormalTok{(Reformat\_subway\_df)}
\end{Highlighting}
\end{Shaded}

\begin{verbatim}
## # A tibble: 6 x 11
##   line     station_name station_latitude station_longitude entry vending
##   <chr>    <chr>                   <dbl>             <dbl> <chr> <chr>  
## 1 4 Avenue 25th St                  40.7             -74.0 YES   YES    
## 2 4 Avenue 25th St                  40.7             -74.0 YES   YES    
## 3 4 Avenue 36th St                  40.7             -74.0 YES   YES    
## 4 4 Avenue 36th St                  40.7             -74.0 YES   YES    
## 5 4 Avenue 36th St                  40.7             -74.0 YES   YES    
## 6 4 Avenue 36th St                  40.7             -74.0 YES   YES    
## # i 5 more variables: entrance_type <chr>, ada <lgl>, entry_logical <lgl>,
## #   route_number <chr>, route_name <chr>
\end{verbatim}

\begin{enumerate}
\def\labelenumi{\arabic{enumi}.}
\setcounter{enumi}{1}
\tightlist
\item
  How many distinct stations serve the A train?
\end{enumerate}

\begin{Shaded}
\begin{Highlighting}[]
\CommentTok{\# number of stations which serve A train}
\NormalTok{a\_train\_stations }\OtherTok{=}\NormalTok{ Reformat\_subway\_df }\SpecialCharTok{|\textgreater{}}
  \FunctionTok{filter}\NormalTok{(route\_name }\SpecialCharTok{==} \StringTok{"A"}\NormalTok{) }\SpecialCharTok{|\textgreater{}}
  \FunctionTok{distinct}\NormalTok{(station\_name, line) }\SpecialCharTok{|\textgreater{}}
  \FunctionTok{nrow}\NormalTok{()}
\end{Highlighting}
\end{Shaded}

There are 60 stations serve the A train.

\begin{enumerate}
\def\labelenumi{\arabic{enumi}.}
\setcounter{enumi}{2}
\tightlist
\item
  Of the stations that serve the A train, how many are ADA compliant?
\end{enumerate}

\begin{Shaded}
\begin{Highlighting}[]
\CommentTok{\# number of ADA compliant stations which serve A train}
\NormalTok{a\_train\_stations\_ADA }\OtherTok{=}\NormalTok{ Reformat\_subway\_df }\SpecialCharTok{|\textgreater{}}
  \FunctionTok{filter}\NormalTok{(route\_name }\SpecialCharTok{==} \StringTok{"A"}\NormalTok{, ada }\SpecialCharTok{==} \ConstantTok{TRUE}\NormalTok{) }\SpecialCharTok{|\textgreater{}}
  \FunctionTok{distinct}\NormalTok{(station\_name, line) }\SpecialCharTok{|\textgreater{}}
  \FunctionTok{nrow}\NormalTok{()}
\end{Highlighting}
\end{Shaded}

There are 17 ADA compliant stations serve the A train.

\subsection{II. Problem 2}\label{ii.-problem-2}

\subsubsection{2.1 Load the Mr.~Trash Wheel
Sheet}\label{load-the-mr.-trash-wheel-sheet}

\begin{itemize}
\tightlist
\item
  Import the \textbf{Mr.~Trash Wheel} sheet, while omitting non-data
  entries;
\item
  Omit rows that do not include dumpster-specific data;
\item
  Round the number of \texttt{sports\_balls}.
\end{itemize}

\begin{Shaded}
\begin{Highlighting}[]
\CommentTok{\# Load the Trash Wheel xlsx}
\NormalTok{trash\_wheel\_path }\OtherTok{=} \StringTok{"./202409 Trash Wheel Collection Data.xlsx"}
\NormalTok{MTW }\OtherTok{=} 
\NormalTok{  readxl}\SpecialCharTok{::}\FunctionTok{read\_excel}\NormalTok{(trash\_wheel\_path, }\AttributeTok{sheet =} \StringTok{"Mr. Trash Wheel"}\NormalTok{, }
                     \AttributeTok{skip =} \DecValTok{1}\NormalTok{, }\AttributeTok{na =} \FunctionTok{c}\NormalTok{(}\StringTok{"NA"}\NormalTok{, }\StringTok{"."}\NormalTok{, }\StringTok{""}\NormalTok{)) }\SpecialCharTok{|\textgreater{}}
\NormalTok{  janitor}\SpecialCharTok{::}\FunctionTok{clean\_names}\NormalTok{() }\SpecialCharTok{|\textgreater{}}
  \FunctionTok{select}\NormalTok{(dumpster}\SpecialCharTok{:}\NormalTok{homes\_powered) }\SpecialCharTok{|\textgreater{}}
  \FunctionTok{drop\_na}\NormalTok{(dumpster) }\SpecialCharTok{|\textgreater{}}
  \FunctionTok{mutate}\NormalTok{(}
    \AttributeTok{sports\_balls\_integer =} \FunctionTok{as.integer}\NormalTok{(}\FunctionTok{round}\NormalTok{(sports\_balls))}
\NormalTok{    )}
\end{Highlighting}
\end{Shaded}

Similarly, import the \textbf{Professor Trash Wheel} and \textbf{Gwynnda
Trash Wheel} sheets.

\begin{Shaded}
\begin{Highlighting}[]
\CommentTok{\# Load the PTW and GTW sheet}
\NormalTok{PTW }\OtherTok{=} 
\NormalTok{  readxl}\SpecialCharTok{::}\FunctionTok{read\_excel}\NormalTok{(trash\_wheel\_path, }\AttributeTok{sheet =} \StringTok{"Professor Trash Wheel"}\NormalTok{, }
                     \AttributeTok{skip =} \DecValTok{1}\NormalTok{, }\AttributeTok{na =} \FunctionTok{c}\NormalTok{(}\StringTok{"NA"}\NormalTok{, }\StringTok{"."}\NormalTok{, }\StringTok{""}\NormalTok{)) }\SpecialCharTok{|\textgreater{}}
\NormalTok{  janitor}\SpecialCharTok{::}\FunctionTok{clean\_names}\NormalTok{() }\SpecialCharTok{|\textgreater{}}
  \FunctionTok{select}\NormalTok{(dumpster}\SpecialCharTok{:}\NormalTok{homes\_powered) }\SpecialCharTok{|\textgreater{}}
  \FunctionTok{drop\_na}\NormalTok{(dumpster, month) }

\NormalTok{GTW }\OtherTok{=} 
\NormalTok{  readxl}\SpecialCharTok{::}\FunctionTok{read\_excel}\NormalTok{(trash\_wheel\_path, }\AttributeTok{sheet =} \StringTok{"Gwynnda Trash Wheel"}\NormalTok{, }
                     \AttributeTok{skip =} \DecValTok{1}\NormalTok{, }\AttributeTok{na =} \FunctionTok{c}\NormalTok{(}\StringTok{"NA"}\NormalTok{, }\StringTok{"."}\NormalTok{, }\StringTok{""}\NormalTok{)) }\SpecialCharTok{|\textgreater{}}
\NormalTok{  janitor}\SpecialCharTok{::}\FunctionTok{clean\_names}\NormalTok{() }\SpecialCharTok{|\textgreater{}}
  \FunctionTok{select}\NormalTok{(dumpster}\SpecialCharTok{:}\NormalTok{homes\_powered) }\SpecialCharTok{|\textgreater{}}
  \FunctionTok{drop\_na}\NormalTok{(dumpster, month) }
\end{Highlighting}
\end{Shaded}

\subsubsection{2.2 Combine PTW and GTW with
MTW}\label{combine-ptw-and-gtw-with-mtw}

\begin{Shaded}
\begin{Highlighting}[]
\NormalTok{MTW }\OtherTok{=}\NormalTok{ MTW }\SpecialCharTok{|\textgreater{}}
  \FunctionTok{mutate}\NormalTok{(}\AttributeTok{category =} \StringTok{"Mr.\_Trash\_Wheel"}\NormalTok{) }\SpecialCharTok{|\textgreater{}}
  \FunctionTok{mutate}\NormalTok{(}\AttributeTok{year =} \FunctionTok{as.character}\NormalTok{(year))}
\NormalTok{PTW }\OtherTok{=}\NormalTok{ PTW }\SpecialCharTok{|\textgreater{}}
  \FunctionTok{mutate}\NormalTok{(}\AttributeTok{category =} \StringTok{"Professor\_Trash\_Wheel"}\NormalTok{) }\SpecialCharTok{|\textgreater{}}
  \FunctionTok{mutate}\NormalTok{(}\AttributeTok{year =} \FunctionTok{as.character}\NormalTok{(year))}
\NormalTok{GTW }\OtherTok{=}\NormalTok{ GTW }\SpecialCharTok{|\textgreater{}}
  \FunctionTok{mutate}\NormalTok{(}\AttributeTok{category =} \StringTok{"Gwynnda\_Trash\_Wheel"}\NormalTok{) }\SpecialCharTok{|\textgreater{}}
  \FunctionTok{mutate}\NormalTok{(}\AttributeTok{year =} \FunctionTok{as.character}\NormalTok{(year))}

\NormalTok{trash\_wheel\_df }\OtherTok{=} 
  \FunctionTok{bind\_rows}\NormalTok{(MTW, PTW, GTW) }\SpecialCharTok{|\textgreater{}}
  \FunctionTok{relocate}\NormalTok{(category) }
\end{Highlighting}
\end{Shaded}

\begin{itemize}
\tightlist
\item
  The number of observations in the resulting datasets is as follows,
  where \texttt{trash\_wheel\_df} represents the final merged dataset:
\end{itemize}

\begin{longtable}[]{@{}
  >{\centering\arraybackslash}p{(\columnwidth - 8\tabcolsep) * \real{0.1667}}
  >{\centering\arraybackslash}p{(\columnwidth - 8\tabcolsep) * \real{0.1875}}
  >{\centering\arraybackslash}p{(\columnwidth - 8\tabcolsep) * \real{0.2396}}
  >{\centering\arraybackslash}p{(\columnwidth - 8\tabcolsep) * \real{0.2188}}
  >{\centering\arraybackslash}p{(\columnwidth - 8\tabcolsep) * \real{0.1875}}@{}}
\toprule\noalign{}
\begin{minipage}[b]{\linewidth}\centering
\end{minipage} & \begin{minipage}[b]{\linewidth}\centering
Mr.~Trash Wheel
\end{minipage} & \begin{minipage}[b]{\linewidth}\centering
Professor Trash Wheel
\end{minipage} & \begin{minipage}[b]{\linewidth}\centering
Gwynnda Trash Wheel
\end{minipage} & \begin{minipage}[b]{\linewidth}\centering
trash\_wheel\_df
\end{minipage} \\
\midrule\noalign{}
\endhead
\bottomrule\noalign{}
\endlastfoot
\textbf{observation} & 651 & 118 & 263 & 1032 \\
\textbf{variable} & 16 & 14 & 13 & 16 \\
\end{longtable}

Compared to \textbf{Mr.~Trash Wheel}, \textbf{Professor Trash Wheel} and
\textbf{Gwynnda Trash Wheel} are missing \texttt{sports\_balls} and
\texttt{glass\_bottles} \& \texttt{sports\_balls}, respectively. These
missing values result in a large number of ``NA'' in the
\textbf{trash\_wheel\_df}. But they are still valid data that represents
their own category, so \textbf{trash\_wheel\_df} is tidy. In addition,
only \texttt{sports\_balls} and \texttt{homes\_powered} trash exist as
multiple decimal places, while the values of other garbage types are
integers.

\begin{Shaded}
\begin{Highlighting}[]
\CommentTok{\# Total weight of trash collected by Professor Trash Wheel}
\NormalTok{PTW\_weight }\OtherTok{=}\NormalTok{ PTW }\SpecialCharTok{|\textgreater{}}
  \FunctionTok{summarise}\NormalTok{(}\AttributeTok{total\_weight =} \FunctionTok{sum}\NormalTok{(weight\_tons, }\AttributeTok{na.rm =} \ConstantTok{TRUE}\NormalTok{))}

\CommentTok{\# Total number of cigarette butts collected by Gwynnda in June 2022}
\NormalTok{GTW\_cigarette }\OtherTok{=}\NormalTok{ GTW }\SpecialCharTok{|\textgreater{}}
  \FunctionTok{filter}\NormalTok{(}\FunctionTok{month}\NormalTok{(date) }\SpecialCharTok{==} \DecValTok{6}\NormalTok{, }
         \FunctionTok{year}\NormalTok{(date) }\SpecialCharTok{==} \DecValTok{2022}\NormalTok{) }\SpecialCharTok{|\textgreater{}}
  \FunctionTok{summarise}\NormalTok{(}\AttributeTok{total\_cigarette\_butts =} \FunctionTok{sum}\NormalTok{(cigarette\_butts, }\AttributeTok{na.rm =} \ConstantTok{TRUE}\NormalTok{))}
\end{Highlighting}
\end{Shaded}

\begin{itemize}
\item
  The total weight of trash collected by Professor Trash Wheel is
  246.74.
\item
  The total number of cigarette butts collected by Gwynnda in June of
  2022 is 18120.
\end{itemize}

\subsection{III. Problem 3}\label{iii.-problem-3}

\subsubsection{3.1 Load all 4 csv}\label{load-all-4-csv}

\textbf{bakers\_df:}

\begin{itemize}
\item
  Load bakers.csv;
\item
  Split the player's first name from \texttt{bakers\_name} as
  \texttt{baker} so that it can be used as a key for later dataset
  merging;
\item
  Ensure there are no duplicate bakers;
\end{itemize}

\begin{Shaded}
\begin{Highlighting}[]
\NormalTok{bakers\_df }\OtherTok{=} \FunctionTok{read.csv}\NormalTok{(}\StringTok{"./gbb\_datasets/bakers.csv"}\NormalTok{, }
                     \AttributeTok{na =} \FunctionTok{c}\NormalTok{(}\StringTok{"NA"}\NormalTok{, }\StringTok{"."}\NormalTok{, }\StringTok{""}\NormalTok{)) }\SpecialCharTok{|\textgreater{}}
\NormalTok{  janitor}\SpecialCharTok{::}\FunctionTok{clean\_names}\NormalTok{() }\SpecialCharTok{|\textgreater{}}
  \FunctionTok{mutate}\NormalTok{(}\AttributeTok{baker =} \FunctionTok{sub}\NormalTok{(}\StringTok{" .*"}\NormalTok{, }\StringTok{" "}\NormalTok{, baker\_name)) }\SpecialCharTok{|\textgreater{}}
  \FunctionTok{mutate}\NormalTok{(}\AttributeTok{baker =} \FunctionTok{iconv}\NormalTok{(baker, }\AttributeTok{from =} \StringTok{"latin1"}\NormalTok{, }\AttributeTok{to =} \StringTok{"UTF{-}8"}\NormalTok{, }\AttributeTok{sub =} \StringTok{""}\NormalTok{)) }\SpecialCharTok{|\textgreater{}}
  \FunctionTok{mutate}\NormalTok{(}\AttributeTok{baker =} \FunctionTok{trimws}\NormalTok{(baker)) }\SpecialCharTok{|\textgreater{}} \CommentTok{\# remove " "}
  \FunctionTok{distinct}\NormalTok{() }\SpecialCharTok{|\textgreater{}}
  \FunctionTok{arrange}\NormalTok{(baker) }\SpecialCharTok{|\textgreater{}}
  \FunctionTok{relocate}\NormalTok{(baker)}
\end{Highlighting}
\end{Shaded}

\textbf{bakes\_df:}

\begin{itemize}
\item
  Load bakes.csv;
\item
  Ensure there are no duplicate bakers;
\item
  Noticed that the name format of the player \texttt{"Jo"} is
  inconsistent with that of other players, since the double quotation
  marks are added. Modify it with \texttt{casematch}.
\end{itemize}

\begin{Shaded}
\begin{Highlighting}[]
\NormalTok{bakes\_df }\OtherTok{=} \FunctionTok{read.csv}\NormalTok{(}\StringTok{"./gbb\_datasets/bakes.csv"}\NormalTok{, }
                     \AttributeTok{na =} \FunctionTok{c}\NormalTok{(}\StringTok{"NA"}\NormalTok{, }\StringTok{"."}\NormalTok{, }\StringTok{""}\NormalTok{)) }\SpecialCharTok{|\textgreater{}}
\NormalTok{  janitor}\SpecialCharTok{::}\FunctionTok{clean\_names}\NormalTok{() }\SpecialCharTok{|\textgreater{}}
  \FunctionTok{distinct}\NormalTok{() }\SpecialCharTok{|\textgreater{}}
  \FunctionTok{mutate}\NormalTok{(}\AttributeTok{baker =} \FunctionTok{case\_match}\NormalTok{(}
\NormalTok{    baker,}
    \StringTok{\textquotesingle{}"Jo"\textquotesingle{}} \SpecialCharTok{\textasciitilde{}} \StringTok{"Jo"}\NormalTok{,}
    \AttributeTok{.default =}\NormalTok{ baker)}
\NormalTok{    ) }\SpecialCharTok{|\textgreater{}} \CommentTok{\#keep other values unchanged}
  \FunctionTok{mutate}\NormalTok{(}\AttributeTok{baker =} \FunctionTok{iconv}\NormalTok{(baker, }\AttributeTok{from =} \StringTok{"latin1"}\NormalTok{, }\AttributeTok{to =} \StringTok{"UTF{-}8"}\NormalTok{, }\AttributeTok{sub =} \StringTok{""}\NormalTok{)) }\SpecialCharTok{|\textgreater{}}
  \FunctionTok{mutate}\NormalTok{(}\AttributeTok{baker =} \FunctionTok{trimws}\NormalTok{(baker)) }\SpecialCharTok{|\textgreater{}}  \CommentTok{\# remove " "}
  \FunctionTok{arrange}\NormalTok{(baker) }\SpecialCharTok{|\textgreater{}}
  \FunctionTok{relocate}\NormalTok{(baker)}
\end{Highlighting}
\end{Shaded}

\textbf{results\_df:}

\begin{itemize}
\tightlist
\item
  Load results.csv;
\end{itemize}

\begin{Shaded}
\begin{Highlighting}[]
\NormalTok{results\_df }\OtherTok{=} \FunctionTok{read.csv}\NormalTok{(}\StringTok{"./gbb\_datasets/results.csv"}\NormalTok{, }\AttributeTok{skip =} \DecValTok{2}\NormalTok{,}
                      \AttributeTok{na =} \FunctionTok{c}\NormalTok{(}\StringTok{"NA"}\NormalTok{, }\StringTok{"."}\NormalTok{, }\StringTok{""}\NormalTok{)) }\SpecialCharTok{|\textgreater{}}
  \FunctionTok{mutate}\NormalTok{(}\AttributeTok{baker =} \FunctionTok{iconv}\NormalTok{(baker, }\AttributeTok{from =} \StringTok{"latin1"}\NormalTok{, }\AttributeTok{to =} \StringTok{"UTF{-}8"}\NormalTok{, }\AttributeTok{sub =} \StringTok{""}\NormalTok{)) }\SpecialCharTok{|\textgreater{}}
  \FunctionTok{mutate}\NormalTok{(}\AttributeTok{baker =} \FunctionTok{trimws}\NormalTok{(baker)) }\SpecialCharTok{|\textgreater{}}  \CommentTok{\# remove " "}
\NormalTok{  janitor}\SpecialCharTok{::}\FunctionTok{clean\_names}\NormalTok{() }
\end{Highlighting}
\end{Shaded}

\subsubsection{3.2 Check the Completeness}\label{check-the-completeness}

\begin{itemize}
\tightlist
\item
  Identify if any baker in \texttt{results\_df} is missing from the
  \texttt{bakers\_df}.
\end{itemize}

\begin{Shaded}
\begin{Highlighting}[]
\CommentTok{\#anti\_join(x, y, by = "key")  x have while y donot have}
\NormalTok{missing\_bakers }\OtherTok{=} \FunctionTok{anti\_join}\NormalTok{(results\_df, bakers\_df, }\AttributeTok{by =} \StringTok{"baker"}\NormalTok{) }

\NormalTok{missing\_bakers}
\end{Highlighting}
\end{Shaded}

\begin{verbatim}
##   series episode  baker technical     result
## 1      2       1 Joanne        11         IN
## 2      2       2 Joanne        10         IN
## 3      2       3 Joanne         1         IN
## 4      2       4 Joanne         8         IN
## 5      2       5 Joanne         6         IN
## 6      2       6 Joanne         1 STAR BAKER
## 7      2       7 Joanne         3         IN
## 8      2       8 Joanne         1     WINNER
\end{verbatim}

The results show that Joanne's series 2 episodes 1-6 is present in the
\texttt{results\_df}, but not in the \texttt{bakers\_df}.

\begin{itemize}
\tightlist
\item
  Identify if any baker's bake in \texttt{results\_df} is missing from
  the \texttt{bakes\_df}.
\end{itemize}

\begin{Shaded}
\begin{Highlighting}[]
\NormalTok{missing\_bakes }\OtherTok{=} \FunctionTok{anti\_join}\NormalTok{(results\_df, bakes\_df, }\AttributeTok{by =} \FunctionTok{c}\NormalTok{(}\StringTok{"baker"}\NormalTok{, }\StringTok{"episode"}\NormalTok{)) }
\FunctionTok{summary}\NormalTok{(missing\_bakes)}
\end{Highlighting}
\end{Shaded}

\begin{verbatim}
##      series          episode          baker             technical     
##  Min.   : 1.000   Min.   : 1.000   Length:554         Min.   : 1.000  
##  1st Qu.: 4.000   1st Qu.: 4.000   Class :character   1st Qu.: 3.000  
##  Median : 7.000   Median : 7.000   Mode  :character   Median : 5.000  
##  Mean   : 6.699   Mean   : 6.377                      Mean   : 5.075  
##  3rd Qu.: 9.000   3rd Qu.: 9.000                      3rd Qu.: 7.000  
##  Max.   :10.000   Max.   :10.000                      Max.   :13.000  
##                                                       NA's   :408     
##     result         
##  Length:554        
##  Class :character  
##  Mode  :character  
##                    
##                    
##                    
## 
\end{verbatim}

The results shows that 84 bakers' bakes are missing from the
\texttt{bakes\_df}.

\subsubsection{3.3 Merge Datasets}\label{merge-datasets}

\begin{Shaded}
\begin{Highlighting}[]
\CommentTok{\# Merge all 3 datasets}
\NormalTok{combined\_df }\OtherTok{=} 
\NormalTok{  results\_df }\SpecialCharTok{|\textgreater{}}
  \FunctionTok{left\_join}\NormalTok{(bakers\_df, }\AttributeTok{by =} \FunctionTok{c}\NormalTok{(}\StringTok{"baker"}\NormalTok{, }\StringTok{"series"}\NormalTok{)) }\SpecialCharTok{|\textgreater{}}
  \FunctionTok{left\_join}\NormalTok{(bakes\_df, }\AttributeTok{by =} \FunctionTok{c}\NormalTok{(}\StringTok{"baker"}\NormalTok{, }\StringTok{"series"}\NormalTok{, }\StringTok{"episode"}\NormalTok{))}

\CommentTok{\# Reorganize the variables to be meaningful}
\NormalTok{final\_df }\OtherTok{=}
\NormalTok{  combined\_df }\SpecialCharTok{|\textgreater{}}
  \FunctionTok{select}\NormalTok{(series, episode, baker\_name, technical, result, signature\_bake, show\_stopper, }
\NormalTok{         baker\_age, hometown, baker\_occupation) }\SpecialCharTok{|\textgreater{}}
  \FunctionTok{arrange}\NormalTok{(series, episode,technical)}

\CommentTok{\# export the final\_df as csv}
\FunctionTok{write\_csv}\NormalTok{(final\_df, }\StringTok{"./gbb\_datasets/Great British Bake Off.csv"}\NormalTok{)}
\end{Highlighting}
\end{Shaded}

The final dataset \texttt{final\_df} has 1136 observations and 10
variables.

In line with the preferences of viewers, this article places The Show's
\texttt{series} and \texttt{eposide} at the top of the dataset, followed
by \texttt{Bakers\textquotesingle{}\ Name} and their \texttt{technical}.
The following are personal characteristics and background information
about each baker, including \texttt{signature\ bake},
\texttt{show\ stopper\ bake}, \texttt{age}, \texttt{hometown}, and
\texttt{occupation\ status}.

\subsubsection{3.4 Star Bakers}\label{star-bakers}

Filter results for Seasons 5 to 10 and select Star Baker.

\begin{Shaded}
\begin{Highlighting}[]
\NormalTok{star\_baker\_df }\OtherTok{=}\NormalTok{ results\_df }\SpecialCharTok{|\textgreater{}}
  \FunctionTok{filter}\NormalTok{(series }\SpecialCharTok{\textgreater{}=} \DecValTok{5} \SpecialCharTok{\&}\NormalTok{ series }\SpecialCharTok{\textless{}=} \DecValTok{10}\NormalTok{, result }\SpecialCharTok{\%in\%} \FunctionTok{c}\NormalTok{(}\StringTok{"STAR BAKER"}\NormalTok{, }\StringTok{"WINNER"}\NormalTok{)) }\SpecialCharTok{|\textgreater{}}
  \FunctionTok{select}\NormalTok{(series, episode, baker, result) }\SpecialCharTok{|\textgreater{}}
  \FunctionTok{mutate}\NormalTok{(}\AttributeTok{result =} \FunctionTok{case\_match}\NormalTok{(}
\NormalTok{    result,}
    \StringTok{"WINNER"} \SpecialCharTok{\textasciitilde{}} \StringTok{"STAR BAKER"}\NormalTok{,}
    \AttributeTok{.default =}\NormalTok{ result)}
\NormalTok{  ) }
\end{Highlighting}
\end{Shaded}

Create a table to show star bakers in Season 5 to 10, organizing by
series and episode.

\begin{Shaded}
\begin{Highlighting}[]
\FunctionTok{library}\NormalTok{(knitr)}
\FunctionTok{library}\NormalTok{(kableExtra)}
\end{Highlighting}
\end{Shaded}

\begin{verbatim}
## The following object is masked from 'package:dplyr':
## 
##     group_rows
\end{verbatim}

\begin{Shaded}
\begin{Highlighting}[]
\NormalTok{star\_baker\_df }\SpecialCharTok{|\textgreater{}}
  \FunctionTok{arrange}\NormalTok{(series, episode) }\SpecialCharTok{|\textgreater{}}
  \FunctionTok{kable}\NormalTok{(}\AttributeTok{caption =} \StringTok{"Star Baker and Winners for Seasons 5 to 10"}\NormalTok{,}
        \AttributeTok{booktabs =} \ConstantTok{TRUE}\NormalTok{) }\SpecialCharTok{|\textgreater{}}
  \FunctionTok{kable\_styling}\NormalTok{() }\SpecialCharTok{|\textgreater{}}
  \FunctionTok{row\_spec}\NormalTok{(}\FunctionTok{which}\NormalTok{(star\_baker\_df}\SpecialCharTok{$}\NormalTok{series }\SpecialCharTok{\%in\%} \FunctionTok{c}\NormalTok{(}\DecValTok{5}\NormalTok{, }\DecValTok{7}\NormalTok{, }\DecValTok{9}\NormalTok{)), }\AttributeTok{background =} \StringTok{"lightgray"}\NormalTok{)}
\end{Highlighting}
\end{Shaded}

\begin{longtable}[t]{rrll}
\caption{\label{tab:unnamed-chunk-19}Star Baker and Winners for Seasons 5 to 10}\\
\toprule
series & episode & baker & result\\
\midrule
\cellcolor{lightgray}{5} & \cellcolor{lightgray}{1} & \cellcolor{lightgray}{Nancy} & \cellcolor{lightgray}{STAR BAKER}\\
\cellcolor{lightgray}{5} & \cellcolor{lightgray}{2} & \cellcolor{lightgray}{Richard} & \cellcolor{lightgray}{STAR BAKER}\\
\cellcolor{lightgray}{5} & \cellcolor{lightgray}{3} & \cellcolor{lightgray}{Luis} & \cellcolor{lightgray}{STAR BAKER}\\
\cellcolor{lightgray}{5} & \cellcolor{lightgray}{4} & \cellcolor{lightgray}{Richard} & \cellcolor{lightgray}{STAR BAKER}\\
\cellcolor{lightgray}{5} & \cellcolor{lightgray}{5} & \cellcolor{lightgray}{Kate} & \cellcolor{lightgray}{STAR BAKER}\\
\addlinespace
\cellcolor{lightgray}{5} & \cellcolor{lightgray}{6} & \cellcolor{lightgray}{Chetna} & \cellcolor{lightgray}{STAR BAKER}\\
\cellcolor{lightgray}{5} & \cellcolor{lightgray}{7} & \cellcolor{lightgray}{Richard} & \cellcolor{lightgray}{STAR BAKER}\\
\cellcolor{lightgray}{5} & \cellcolor{lightgray}{8} & \cellcolor{lightgray}{Richard} & \cellcolor{lightgray}{STAR BAKER}\\
\cellcolor{lightgray}{5} & \cellcolor{lightgray}{9} & \cellcolor{lightgray}{Richard} & \cellcolor{lightgray}{STAR BAKER}\\
\cellcolor{lightgray}{5} & \cellcolor{lightgray}{10} & \cellcolor{lightgray}{Nancy} & \cellcolor{lightgray}{STAR BAKER}\\
\addlinespace
6 & 1 & Marie & STAR BAKER\\
6 & 2 & Ian & STAR BAKER\\
6 & 3 & Ian & STAR BAKER\\
6 & 4 & Ian & STAR BAKER\\
6 & 5 & Nadiya & STAR BAKER\\
\addlinespace
6 & 6 & Mat & STAR BAKER\\
6 & 7 & Tamal & STAR BAKER\\
6 & 8 & Nadiya & STAR BAKER\\
6 & 9 & Nadiya & STAR BAKER\\
6 & 10 & Nadiya & STAR BAKER\\
\addlinespace
\cellcolor{lightgray}{7} & \cellcolor{lightgray}{1} & \cellcolor{lightgray}{Jane} & \cellcolor{lightgray}{STAR BAKER}\\
\cellcolor{lightgray}{7} & \cellcolor{lightgray}{2} & \cellcolor{lightgray}{Candice} & \cellcolor{lightgray}{STAR BAKER}\\
\cellcolor{lightgray}{7} & \cellcolor{lightgray}{3} & \cellcolor{lightgray}{Tom} & \cellcolor{lightgray}{STAR BAKER}\\
\cellcolor{lightgray}{7} & \cellcolor{lightgray}{4} & \cellcolor{lightgray}{Benjamina} & \cellcolor{lightgray}{STAR BAKER}\\
\cellcolor{lightgray}{7} & \cellcolor{lightgray}{5} & \cellcolor{lightgray}{Candice} & \cellcolor{lightgray}{STAR BAKER}\\
\addlinespace
\cellcolor{lightgray}{7} & \cellcolor{lightgray}{6} & \cellcolor{lightgray}{Tom} & \cellcolor{lightgray}{STAR BAKER}\\
\cellcolor{lightgray}{7} & \cellcolor{lightgray}{7} & \cellcolor{lightgray}{Andrew} & \cellcolor{lightgray}{STAR BAKER}\\
\cellcolor{lightgray}{7} & \cellcolor{lightgray}{8} & \cellcolor{lightgray}{Candice} & \cellcolor{lightgray}{STAR BAKER}\\
\cellcolor{lightgray}{7} & \cellcolor{lightgray}{9} & \cellcolor{lightgray}{Andrew} & \cellcolor{lightgray}{STAR BAKER}\\
\cellcolor{lightgray}{7} & \cellcolor{lightgray}{10} & \cellcolor{lightgray}{Candice} & \cellcolor{lightgray}{STAR BAKER}\\
\addlinespace
8 & 1 & Steven & STAR BAKER\\
8 & 2 & Steven & STAR BAKER\\
8 & 3 & Julia & STAR BAKER\\
8 & 4 & Kate & STAR BAKER\\
8 & 5 & Sophie & STAR BAKER\\
\addlinespace
8 & 6 & Liam & STAR BAKER\\
8 & 7 & Steven & STAR BAKER\\
8 & 8 & Stacey & STAR BAKER\\
8 & 9 & Sophie & STAR BAKER\\
8 & 10 & Sophie & STAR BAKER\\
\addlinespace
\cellcolor{lightgray}{9} & \cellcolor{lightgray}{1} & \cellcolor{lightgray}{Manon} & \cellcolor{lightgray}{STAR BAKER}\\
\cellcolor{lightgray}{9} & \cellcolor{lightgray}{2} & \cellcolor{lightgray}{Rahul} & \cellcolor{lightgray}{STAR BAKER}\\
\cellcolor{lightgray}{9} & \cellcolor{lightgray}{3} & \cellcolor{lightgray}{Rahul} & \cellcolor{lightgray}{STAR BAKER}\\
\cellcolor{lightgray}{9} & \cellcolor{lightgray}{4} & \cellcolor{lightgray}{Dan} & \cellcolor{lightgray}{STAR BAKER}\\
\cellcolor{lightgray}{9} & \cellcolor{lightgray}{5} & \cellcolor{lightgray}{Kim-Joy} & \cellcolor{lightgray}{STAR BAKER}\\
\addlinespace
\cellcolor{lightgray}{9} & \cellcolor{lightgray}{6} & \cellcolor{lightgray}{Briony} & \cellcolor{lightgray}{STAR BAKER}\\
\cellcolor{lightgray}{9} & \cellcolor{lightgray}{7} & \cellcolor{lightgray}{Kim-Joy} & \cellcolor{lightgray}{STAR BAKER}\\
\cellcolor{lightgray}{9} & \cellcolor{lightgray}{8} & \cellcolor{lightgray}{Ruby} & \cellcolor{lightgray}{STAR BAKER}\\
\cellcolor{lightgray}{9} & \cellcolor{lightgray}{9} & \cellcolor{lightgray}{Ruby} & \cellcolor{lightgray}{STAR BAKER}\\
\cellcolor{lightgray}{9} & \cellcolor{lightgray}{10} & \cellcolor{lightgray}{Rahul} & \cellcolor{lightgray}{STAR BAKER}\\
\addlinespace
10 & 1 & Michelle & STAR BAKER\\
10 & 2 & Alice & STAR BAKER\\
10 & 3 & Michael & STAR BAKER\\
10 & 4 & Steph & STAR BAKER\\
10 & 5 & Steph & STAR BAKER\\
\addlinespace
10 & 6 & Steph & STAR BAKER\\
10 & 7 & Henry & STAR BAKER\\
10 & 8 & Steph & STAR BAKER\\
10 & 9 & Alice & STAR BAKER\\
10 & 10 & David & STAR BAKER\\
\bottomrule
\end{longtable}

\begin{Shaded}
\begin{Highlighting}[]
\NormalTok{baker\_frequency }\OtherTok{=}\NormalTok{ star\_baker\_df }\SpecialCharTok{|\textgreater{}}
  \FunctionTok{count}\NormalTok{(baker, }\AttributeTok{sort =} \ConstantTok{TRUE}\NormalTok{)  }\CommentTok{\# sort = TRUE means descending}
\FunctionTok{head}\NormalTok{(baker\_frequency)}
\end{Highlighting}
\end{Shaded}

\begin{verbatim}
##     baker n
## 1 Richard 5
## 2 Candice 4
## 3  Nadiya 4
## 4   Steph 4
## 5     Ian 3
## 6   Rahul 3
\end{verbatim}

\begin{itemize}
\item
  \textbf{Predictable Overall Winners:} \emph{Richard Burr} won STAR
  BAKER 5 times, which is the most of all bakers. \emph{Candice Brown},
  \emph{Nadiya Hussain} and \emph{Steph Blackwell} won 4 times.
  Additionally, \emph{Richard Burr} from series 5, \emph{Ian Cumming}
  and \emph{Nadiya Hussain} from series 6, \emph{Steph Blackwell} from
  series 10 all consistently achieved STAR BAKER during their own
  series. To summarize, \textbf{Richard Burr} is the most predictable
  overall winners.
\item
  \textbf{Surprises:} It is surprising that \textbf{David Atherton} from
  series 10 was crowned STAR BAKER in episode 10, even though he did not
  won anyone before.
\end{itemize}

\subsubsection{3.5 viewers\_df}\label{viewers_df}

Start by importing viewers.csv.

\begin{Shaded}
\begin{Highlighting}[]
\NormalTok{viewers\_df }\OtherTok{=} \FunctionTok{read.csv}\NormalTok{(}\StringTok{"./gbb\_datasets/viewers.csv"}\NormalTok{, }
                      \AttributeTok{na =} \FunctionTok{c}\NormalTok{(}\StringTok{"NA"}\NormalTok{, }\StringTok{"."}\NormalTok{, }\StringTok{""}\NormalTok{)) }\SpecialCharTok{|\textgreater{}}
\NormalTok{  janitor}\SpecialCharTok{::}\FunctionTok{clean\_names}\NormalTok{() }

\FunctionTok{head}\NormalTok{(viewers\_df, }\DecValTok{10}\NormalTok{)}
\end{Highlighting}
\end{Shaded}

\begin{verbatim}
##    episode series_1 series_2 series_3 series_4 series_5 series_6 series_7
## 1        1     2.24     3.10     3.85     6.60    8.510    11.62    13.58
## 2        2     3.00     3.53     4.60     6.65    8.790    11.59    13.45
## 3        3     3.00     3.82     4.53     7.17    9.280    12.01    13.01
## 4        4     2.60     3.60     4.71     6.82   10.250    12.36    13.29
## 5        5     3.03     3.83     4.61     6.95    9.950    12.39    13.12
## 6        6     2.75     4.25     4.82     7.32   10.130    12.00    13.13
## 7        7       NA     4.42     5.10     7.76   10.280    12.35    13.45
## 8        8       NA     5.06     5.35     7.41    9.023    11.09    13.26
## 9        9       NA       NA     5.70     7.41   10.670    12.65    13.44
## 10      10       NA       NA     6.74     9.45   13.510    15.05    15.90
##    series_8 series_9 series_10
## 1      9.46     9.55      9.62
## 2      9.23     9.31      9.38
## 3      8.68     8.91      8.94
## 4      8.55     8.88      8.96
## 5      8.61     8.67      9.26
## 6      8.61     8.91      8.70
## 7      9.01     9.22      8.98
## 8      8.95     9.69      9.19
## 9      9.03     9.50      9.34
## 10    10.04    10.34     10.05
\end{verbatim}

\begin{Shaded}
\begin{Highlighting}[]
\NormalTok{average\_series\_1 }\OtherTok{=}\NormalTok{ viewers\_df }\SpecialCharTok{|\textgreater{}}
  \FunctionTok{summarize}\NormalTok{(}\AttributeTok{average =} \FunctionTok{mean}\NormalTok{(series\_1, }\AttributeTok{na.rm =} \ConstantTok{TRUE}\NormalTok{))}
\NormalTok{average\_series\_5 }\OtherTok{=}\NormalTok{ viewers\_df }\SpecialCharTok{|\textgreater{}}
  \FunctionTok{summarize}\NormalTok{(}\AttributeTok{average =} \FunctionTok{mean}\NormalTok{(series\_5, }\AttributeTok{na.rm =} \ConstantTok{TRUE}\NormalTok{))}
\end{Highlighting}
\end{Shaded}

The average viewership in Season 1 is 2.77. In Season 5 is 10.0393.

\end{document}
